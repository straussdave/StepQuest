%
% FH Technikum Wien
% !TEX encoding = UTF-8 Unicode
% LTeX: language=en
%
% Erstellung von Master- und Bachelorarbeiten an der FH Technikum Wien mit Hilfe von LaTeX und der Klasse TWBOOK
%
% Um ein eigenes Dokument zu erstellen, müssen Sie folgendes ergänzen:
% 1) Mit \documentclass[..] einstellen
%    * Master- oder Bachelorarbeit
%      * Master   ... Masterarbeit
%      * Bachelor ... Bachelorarbeit
%    * Sprache (english, german, ngerman)
%    * Zitationsstandard (Harvard, IEEE) (Standard: IEEE)
%    * Biber oder BibTeX als Literaturbackend (Biber, BibTeX) (Standard: Biber)
% 2) Studiengang ausfüllen
% 3) Deckblatt, Kurzfassung, etc. ausfüllen
% 4) und die Arbeit schreiben (die verwendeten Literaturquellen in Literatur.bib eintragen)
%
% Getestet mit TeXstudio mit Zeichenkodierung utf-8 (=ansinew/latin1) und TexLive unter Ubuntu
% Zu beachten ist, dass die Kodierung der Datei mit der Kodierung des paketes inputenc zusammen passt!
% Die Kodierung der Datei twbook.cls MUSS ANSI betragen!
% Bei der Verwendung von UTF8 muss nicht nur die Kodierung des Dokuments auf UTF8 gestellt sein, sondern auch die des BibTex-Files!
%
% Bugreports und Feedback bitte per E-Mail an latex@technikum-wien.at
%
% Version V2.24 von 2024-12-19 otrebski
%
\documentclass[Master,english,IEEE]{twbook}
\usepackage[utf8]{inputenc}
\usepackage[T1]{fontenc}

% bitte setzen Sie hier den Studiengang
\degreecourse{SWE}

\addbibresource{Literatur.bib}
%% Definieren Sie hier bei Bedarf weitere Literaturdatenbanken

% Die nachfolgenden Pakete stellen sonst nicht benötigte Features zur Verfügung

%
% Einträge für Deckblatt, Kurzfassung, etc.
%
\title{Project Report Step Quest}
\author{David Strauß, BSc}
\studentnumber{52112310}
%\author{Titel Vorname Name, Titel\and{}Titel Vorname Name, Titel}
%\studentnumber{XXXXXXXXXXXXXXX\and{}XXXXXXXXXXXXXXX}
\supervisor{Jürgen Konrad, MSc}
\secondsupervisor{Ing. Dr. techn. Dominik Dolezal, MSc}
%\supervisor[Begutachterin]{Titel Vorname Name, Titel}
%\secondsupervisor{Titel Vorname Name, Titel}
%\secondsupervisor[Begutachter]{Titel Vorname Name, Titel}
%\secondsupervisor[Begutachterinnen]{Titel Vorname Name, Titel}
\place{Wien}
%\kurzfassung{\blindtext}
%\schlagworte{Schlagwort1, Schlagwort2, Schlagwort3, Schlagwort4}
%\outline{\blindtext}
%\keywords{Keyword1, Keyword2, Keyword3, Keyword4}
%\acknowledgements{\blindtext}

\begin{document}

\maketitle

%
% .. und hier beginnt die eigentliche Arbeit. Viel Erfolg beim Verfassen!
%

\chapter{Introduction}

\section{Goal}
The goal of StepQuest is to develop a Unity-based mobile game that motivates players to go
for regular walks by linking real-world movement to in-game progress, quest completion,
and unlockable content. The project follows a theory-informed exergame approach, emphasizing that
successful activity-based games must balance exercise goals with enjoyable gameplay and sustainable
player engagement \cite{10.1145/1321261.1321313,Kubota2022Pilot}.

\section{Background}
Physical inactivity is a significant public health concern and is addressed as a global priority in health
promotion strategies \cite{world2019global}. In response, digital interventions, including mobile games that
integrate movement, have gained attention because they can combine entertainment with behavior change
mechanisms in everyday contexts. Exergame research suggests that effectiveness depends not only on
including physical activity, but also on how the game is designed: players need clear goals, meaningful
feedback, appropriate difficulty and progression, and experiences that remain engaging over time
\cite{10.1145/1321261.1321313}.

StepQuest builds on these insights by using walking (step counts) as the core resource for progress and
daily quest-based gameplay. This approach aligns with findings from theory-based exergames: when game
elements are explicitly analyzed through a behavioral lens, designers can better justify which mechanics
should motivate activity and which are expected to drive retention \cite{Kubota2022Pilot}. However, there is
also a risk in relying too strongly on a single gamification model. For example, reflection on the Octalysis
framework highlights that while it is useful for ideation and qualitative evaluation, applying it rigorously
requires careful interpretation and may introduce subjectivity if used without clear operationalization
\cite{Weber2022Reflection}.

All design and development decisions in StepQuest are therefore informed by existing research on
gamification and behavior change. As a primary design framework, the project leans on the Octalysis
gamification model to structure motivational mechanics such as progression, ownership, social influence,
and scarcity-driven rewards \cite{octalysis,key}. In addition, StepQuest considers the Behavior Change Wheel
as a complementary perspective to ensure that the implemented mechanics do not only feel motivating, but
also plausibly support real-world walking behavior (e.g., by supporting capability, opportunity, and
motivation). % Consider adding a canonical BCW citation here in your BibTeX.

\section{Research Questions and Hypothesis}
\textbf{Research Questions:}
\begin{itemize}
  \item \textbf{RQ1:} Does playing StepQuest increase the frequency and duration of walks compared to a control group?
  \item \textbf{RQ2:} How does player engagement with StepQuest evolve over time, and what factors influence sustained use?
\end{itemize}
\textbf{Hypothesis:}
Targeted use of motivational strategies in a mobile game can increase motivation to engage in regular
walking \cite{key,octalysis}. Based on exergame design literature, StepQuest further assumes that engagement
and adherence are more likely when progression and feedback are structured in a way that supports
long-term play rather than short-lived novelty effects \cite{10.1145/1321261.1321313,Kubota2022Pilot}.

\chapter{Methodology}

\section{Design Rationale and Framework Selection}
Before implementation, StepQuest was designed around established principles from exergame research and
behavior change theory. Exergame design literature emphasizes that successful activity-based games must
balance exercise-related goals with enjoyable, safe, and sustainable gameplay (e.g., appropriate feedback,
progression, and minimizing elements that interfere with physical activity) \cite{10.1145/1321261.1321313,Kubota2022Pilot}.

For motivational design, StepQuest uses the Octalysis framework to structure and evaluate game mechanics
according to different motivational drivers \cite{octalysis,key}. Because Octalysis can be interpretive and may
introduce subjectivity if used without clear operationalization, the framework is applied primarily as an ideation
and qualitative evaluation tool rather than as a strict measurement instrument \cite{Weber2022Reflection}.

To complement this, StepQuest also considers the Behaviour Change Wheel (BCW). The BCW links behavior
change intervention functions (e.g., education, incentivisation, persuasion) to determinants of behavior through
the COM-B model (capability, opportunity, motivation), supporting a more explicit justification of how selected
mechanics may influence real-world walking behavior \cite{bcw2011}.

\subsection{Key Pre-development Design Decisions}
The following design decisions were defined before starting development:

\begin{itemize}
  \item \textbf{Low-attention walking experience:} No dialogue or reading-heavy interaction during walks, because
  players are expected to keep attention on the environment while walking. This follows the general exergame
  design principle that gameplay should not undermine safety or the physical activity experience \cite{10.1145/1321261.1321313}.
  \item \textbf{Quest-centered content architecture:} Quest-related dialogue options and narrative snippets are stored
  directly on quest ScriptableObjects to keep content modular and easy to iterate without refactoring the core game logic.
  \item \textbf{Daily choice structure:} Players select one quest target per day from a small set of options, supporting
  a clear “daily loop” and reducing complexity in the prototype.
\end{itemize}

\subsection{Octalysis and BCW Mapping in StepQuest}
StepQuest applies Octalysis as a structured checklist of motivational drivers \cite{octalysis,key}, and maps selected
drivers to BCW intervention functions to make the intended behavior-change mechanisms explicit \cite{bcw2011}:

\begin{itemize}
  \item \textbf{Epic Meaning \& Calling:} Narrative framing (escape the planet by collecting ship parts). \\
  \textit{BCW link:} Persuasion and Education (positive framing and meaning).
  \item \textbf{Development \& Accomplishment:} Progress bars, achievements/badges. \\
  \textit{BCW link:} Incentivisation (rewards) and Reinforcement.
  \item \textbf{Creativity \& Feedback:} Optional goal choice (e.g., step target or time-based target). \\
  \textit{BCW link:} Training (supporting capability through self-set goals) and Enablement.
  \item \textbf{Ownership \& Possession:} Collectible ship parts as a visible collection. \\
  \textit{BCW link:} Incentivisation (collection as reward) and Enablement (progress visibility).
  \item \textbf{Social Influence \& Relatedness (optional):} Low-effort social sharing (“brag button”). \\
  \textit{BCW link:} Modelling (social comparison / exemplars). \\
  \textit{Note:} May be omitted due to prototype scope constraints.
  \item \textbf{Scarcity \& Impatience:} One ship part per day (appointment-like dynamics). \\
  \textit{BCW link:} Incentivisation (time-limited opportunity), but requires fairness/clarity to avoid frustration.
  \item \textbf{Unpredictability \& Curiosity:} Occasional surprise rewards or short quizzes at the end of a walk. \\
  \textit{BCW link:} Education (knowledge prompts) and Incentivisation (bonus rewards).
  \item \textbf{Loss \& Avoidance (used cautiously):} Soft “missed chance” framing rather than punishment. \\
  \textit{BCW link:} Coercion in a mild form (avoid overuse to prevent negative affect).
\end{itemize}

\section{Prototype Development}
The project is developed in Unity targeting Android. Android was chosen as a platform due to its wide
availability and ease of deployment for testing purposes. The game prototype includes core mechanics such as step
tracking, questing, and a simple user interface. The development process follows the waterfall model, due to the nature
of the project as a prototype with a defined scope.

\section{User Testing}
User testing will be conducted with a small group of participants.
Participants will be asked to install the APK on their Android devices and use the app over a period of two weeks.
After the testing period, participants will be surveyed regarding their experience, motivation levels, and feedback on
the app.

\section{Data Collection}
After deployment, data on user engagement and step counts will be collected. For this,
a logging system will be implemented to track relevant metrics. They will be exported after the tester is done playing
via a debug menu option.
\chapter{Results}
\section{Status}

At the current stage, the core gameplay loop is playable end-to-end. The player is greeted by an
in-game companion character (dialogue screen) and proceeds into a daily mission selection flow.
Missions are presented as ship parts with clearly communicated step goals (e.g., 2500--3000 steps).
A day counter is shown in the main UI (``Sol 8''), and the build includes entry points for a settings screen
as well as a collection view.

\begin{itemize}
  \item \textbf{Dialogue onboarding flow:} A character-driven dialogue screen introduces the interaction and
  transitions into the daily choice (``What part would you like to retrieve today?'').
  \item \textbf{Daily mission selection implemented:} The player can choose from multiple available parts
  (e.g., Wing, Cockpit), each associated with a step target.
  \item \textbf{Prototype progression structure:} The chosen mission corresponds to a collectible ship part,
  supporting a clear short-term goal for walking and a longer-term collection objective.
  \item \textbf{Collection/inspection screen:} A dedicated view exists to inspect the ship/collection state
  (currently represented visually as a ship preview and a back-navigation flow).
  \item \textbf{General UX structure in place:} Consistent UI theme, navigation (e.g., back button), and
  a dedicated \textit{Collection} button from the main screen.
\end{itemize}
\textbf{Open work:}
While the loop is functional, the prototype is not feature-complete. Remaining work includes completing
content breadth, refining balancing, adding excitement features such as sounds and animations
and finalizing support features needed for the evaluation phase 
(e.g., robust data export/logging, and polish for settings and collection feedback).

\chapter{Discussion}

\section{Timeline}
The project timeline is structured to complete development by mid-February, allowing time for a personal
test run before starting external user testing. User tests are planned for March - April, followed by evaluation
and thesis writing in April - May.

\begin{itemize}
  \item \textbf{Now - mid February:} Finalize prototype to a ``test-ready'' state.
  \begin{itemize}
    \item Complete remaining core features.
    \item Content completion for the planned missions/parts and balancing of step goals.
    \item Internal QA: bug fixing, usability improvements, and edge-case handling (permissions, background step tracking).
  \end{itemize}

  \item \textbf{Mid February - end February:} Personal test run.
  \begin{itemize}
    \item Run the prototype daily under realistic conditions to validate the gameplay loop.
    \item Verify data collection works reliably over multiple days.
    \item Identify and fix critical issues before involving participants.
  \end{itemize}

  \item \textbf{March - April:} User testing phase.
  \begin{itemize}
    \item Distribute APK to participants and support installation.
    \item Collect quantitative logs and qualitative feedback (survey/interview).
  \end{itemize}

  \item \textbf{April:} Evaluation phase.
  \begin{itemize}
    \item Analyze engagement patterns over time and summarize qualitative feedback.
    \item Answer research questions using collected data.
    \item Derive implications for design improvements and discuss limitations.
  \end{itemize}

  \item \textbf{April - May:} Thesis writing and finalization.
  \begin{itemize}
    \item Integrate evaluation findings with theory (Octalysis/BCW) and prior work.
    \item Final proofreading, formatting, and submission preparation.
  \end{itemize}
\end{itemize}

% Hier können Sie Ihre KI-Tools dokumentieren. Diese werden automatisch in eine Tabelle integriert.
\aitoolentry{DeepL Translate}{Translation of an article in English}{Source (XXX), Chapter X on page X-X}
\aitoolentry{Chat GPT 4.0}{Grammar and Spelling}{"Please list issues with spelling and grammar in the following text: ...", Entire document}

%
% Hier beginnen die Verzeichnisse.
%
\clearpage
\printbibliography
\clearpage

% Das Abbildungsverzeichnis
\listoffigures
\clearpage

% Das Tabellenverzeichnis
\listoftables
\clearpage

% Das Verzeichnis über die verwendeten KI-Tools
\listaitools
\clearpage

\phantomsection
\addcontentsline{toc}{chapter}{\listacroname}
\chapter*{\listacroname}
\begin{acronym}[XXXXX]
    \acro{ABC}[ABC]{Alphabet}
    \acro{WWW}[WWW]{world wide web}
    \acro{ROFL}[ROFL]{Rolling on floor laughing}
\end{acronym}
\end{document}