\chapter{Introduction}

\section{Goal}
The goal of StepQuest is to develop a Unity-based mobile game that motivates players to go
for regular walks by linking real-world movement to in-game progress, quest completion,
and unlockable content. The project follows a theory-informed exergame approach, emphasizing that
successful activity-based games must balance exercise goals with enjoyable gameplay and sustainable
player engagement \cite{10.1145/1321261.1321313,Kubota2022Pilot}.

\section{Background}
Physical inactivity is a significant public health concern and is addressed as a global priority in health
promotion strategies \cite{world2019global}. In response, digital interventions, including mobile games that
integrate movement, have gained attention because they can combine entertainment with behavior change
mechanisms in everyday contexts. Exergame research suggests that effectiveness depends not only on
including physical activity, but also on how the game is designed: players need clear goals, meaningful
feedback, appropriate difficulty and progression, and experiences that remain engaging over time
\cite{10.1145/1321261.1321313}.

StepQuest builds on these insights by using walking (step counts) as the core resource for progress and
daily quest-based gameplay. This approach aligns with findings from theory-based exergames: when game
elements are explicitly analyzed through a behavioral lens, designers can better justify which mechanics
should motivate activity and which are expected to drive retention \cite{Kubota2022Pilot}. However, there is
also a risk in relying too strongly on a single gamification model. For example, reflection on the Octalysis
framework highlights that while it is useful for ideation and qualitative evaluation, applying it rigorously
requires careful interpretation and may introduce subjectivity if used without clear operationalization
\cite{Weber2022Reflection}.

All design and development decisions in StepQuest are therefore informed by existing research on
gamification and behavior change. As a primary design framework, the project leans on the Octalysis
gamification model to structure motivational mechanics such as progression, ownership, social influence,
and scarcity-driven rewards \cite{octalysis,key}. In addition, StepQuest considers the Behavior Change Wheel
as a complementary perspective to ensure that the implemented mechanics do not only feel motivating, but
also plausibly support real-world walking behavior (e.g., by supporting capability, opportunity, and
motivation). % Consider adding a canonical BCW citation here in your BibTeX.

\section{Research Questions and Hypothesis}
\textbf{Research Questions:}
\begin{itemize}
  \item \textbf{RQ1:} Does playing StepQuest increase the frequency and duration of walks compared to a control group?
  \item \textbf{RQ2:} How does player engagement with StepQuest evolve over time, and what factors influence sustained use?
\end{itemize}
\textbf{Hypothesis:}
Targeted use of motivational strategies in a mobile game can increase motivation to engage in regular
walking \cite{key,octalysis}. Based on exergame design literature, StepQuest further assumes that engagement
and adherence are more likely when progression and feedback are structured in a way that supports
long-term play rather than short-lived novelty effects \cite{10.1145/1321261.1321313,Kubota2022Pilot}.
