\chapter{Methodology}

\section{Design Rationale and Framework Selection}
Before implementation, StepQuest was designed around established principles from exergame research and
behavior change theory. Exergame design literature emphasizes that successful activity-based games must
balance exercise-related goals with enjoyable, safe, and sustainable gameplay (e.g., appropriate feedback,
progression, and minimizing elements that interfere with physical activity) \cite{10.1145/1321261.1321313,Kubota2022Pilot}.

For motivational design, StepQuest uses the Octalysis framework to structure and evaluate game mechanics
according to different motivational drivers \cite{octalysis,key}. Because Octalysis can be interpretive and may
introduce subjectivity if used without clear operationalization, the framework is applied primarily as an ideation
and qualitative evaluation tool rather than as a strict measurement instrument \cite{Weber2022Reflection}.

To complement this, StepQuest also considers the Behaviour Change Wheel (BCW). The BCW links behavior
change intervention functions (e.g., education, incentivisation, persuasion) to determinants of behavior through
the COM-B model (capability, opportunity, motivation), supporting a more explicit justification of how selected
mechanics may influence real-world walking behavior \cite{bcw2011}.

\subsection{Key Pre-development Design Decisions}
The following design decisions were defined before starting development:

\begin{itemize}
  \item \textbf{Low-attention walking experience:} No dialogue or reading-heavy interaction during walks, because
  players are expected to keep attention on the environment while walking. This follows the general exergame
  design principle that gameplay should not undermine safety or the physical activity experience \cite{10.1145/1321261.1321313}.
  \item \textbf{Quest-centered content architecture:} Quest-related dialogue options and narrative snippets are stored
  directly on quest ScriptableObjects to keep content modular and easy to iterate without refactoring the core game logic.
  \item \textbf{Daily choice structure:} Players select one quest target per day from a small set of options, supporting
  a clear “daily loop” and reducing complexity in the prototype.
\end{itemize}

\subsection{Octalysis and BCW Mapping in StepQuest}
StepQuest applies Octalysis as a structured checklist of motivational drivers \cite{octalysis,key}, and maps selected
drivers to BCW intervention functions to make the intended behavior-change mechanisms explicit \cite{bcw2011}:

\begin{itemize}
  \item \textbf{Epic Meaning \& Calling:} Narrative framing (escape the planet by collecting ship parts). \\
  \textit{BCW link:} Persuasion and Education (positive framing and meaning).
  \item \textbf{Development \& Accomplishment:} Progress bars, achievements/badges. \\
  \textit{BCW link:} Incentivisation (rewards) and Reinforcement.
  \item \textbf{Creativity \& Feedback:} Optional goal choice (e.g., step target or time-based target). \\
  \textit{BCW link:} Training (supporting capability through self-set goals) and Enablement.
  \item \textbf{Ownership \& Possession:} Collectible ship parts as a visible collection. \\
  \textit{BCW link:} Incentivisation (collection as reward) and Enablement (progress visibility).
  \item \textbf{Social Influence \& Relatedness (optional):} Low-effort social sharing (“brag button”). \\
  \textit{BCW link:} Modelling (social comparison / exemplars). \\
  \textit{Note:} May be omitted due to prototype scope constraints.
  \item \textbf{Scarcity \& Impatience:} One ship part per day (appointment-like dynamics). \\
  \textit{BCW link:} Incentivisation (time-limited opportunity), but requires fairness/clarity to avoid frustration.
  \item \textbf{Unpredictability \& Curiosity:} Occasional surprise rewards or short quizzes at the end of a walk. \\
  \textit{BCW link:} Education (knowledge prompts) and Incentivisation (bonus rewards).
  \item \textbf{Loss \& Avoidance (used cautiously):} Soft “missed chance” framing rather than punishment. \\
  \textit{BCW link:} Coercion in a mild form (avoid overuse to prevent negative affect).
\end{itemize}

\section{Prototype Development}
The project is developed in Unity targeting Android. Android was chosen as a platform due to its wide
availability and ease of deployment for testing purposes. The game prototype includes core mechanics such as step
tracking, questing, and a simple user interface. The development process follows the waterfall model, due to the nature
of the project as a prototype with a defined scope.

\section{User Testing}
User testing will be conducted with a small group of participants.
Participants will be asked to install the APK on their Android devices and use the app over a period of two weeks.
After the testing period, participants will be surveyed regarding their experience, motivation levels, and feedback on
the app.

\section{Data Collection}
After deployment, data on user engagement and step counts will be collected. For this,
a logging system will be implemented to track relevant metrics. They will be exported after the tester is done playing
via a debug menu option.